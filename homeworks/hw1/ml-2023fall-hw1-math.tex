\documentclass{article}
\usepackage[utf8]{inputenc}
\usepackage{enumitem}
\usepackage{ulem}
\usepackage{mathrsfs}
\usepackage{amsmath}
\usepackage{float}
\usepackage{hyperref}
\usepackage{xcolor}
\usepackage{listings}
\usepackage{color}
\usepackage[utf8]{inputenc}
\usepackage{CJK}
\usepackage{amsthm,amsmath,amssymb}
\usepackage{hyperref}
\usepackage{relsize}
\usepackage{graphicx}

\newtheorem{problem}{Problem}

\newcommand{\N}{\mathcal N}
\newcommand{\Y}{\mathbf Y}
\newcommand{\real}{\mathbb{R}}
\newcommand{\X}{\mathbf X}
\newcommand{\Z}{\mathbf Z}

\newcommand*{\tran}{^{\mkern-1.5mu\mathsf{T}}}
\newcommand*{\hermitian}{^{\mkern-1.5mu\mathsf{H}}}

\newcommand{\Tau}{\mathrm{T}}

\DeclareMathOperator{\Tr}{Tr}

\def\Epsilon{E}
\def\Eta{H}

\def\veczero{{\mathbf 0}}
\def\vec1{{\mathbf 1}}
\def\matzero{{\mathbf 0}}

\def\veca{{\mathbf a}}
\def\vecb{{\mathbf b}}
\def\vecc{{\mathbf c}}
\def\vecd{{\mathbf d}}
\def\vece{{\mathbf e}}
\def\vecf{{\mathbf f}}
\def\vecg{{\mathbf g}}
\def\vech{{\mathbf h}}
\def\veck{{\mathbf k}}
\def\vecm{{\mathbf m}}
\def\vecn{{\mathbf n}}
\def\vecp{{\mathbf p}}
\def\vecq{{\mathbf q}}
\def\vecr{{\mathbf r}}
\def\vect{{\mathbf t}}
\def\vecu{{\mathbf u}}
\def\vecv{{\mathbf v}}
\def\vecw{{\mathbf w}}
\def\vecx{{\mathbf x}}
\def\vecy{{\mathbf y}}
\def\vecz{{\mathbf z}}

\def\vecalpha{{\mathbf \alpha}}
\def\vecbeta{{\mathbf \beta}}
\def\vecepsilon{{\boldsymbol \epsilon}}
\def\veclambda{{\mathbf \lambda}}
\def\vecvarepsilon{{\boldsymbol \varepsilon}}
\def\vecgamma{{\boldsymbol \gamma}}
\def\vecmu{{\boldsymbol \mu}}
\def\vecnu{{\boldsymbol \nu}}
\def\vecomega{{\boldsymbol \omega}}
\def\vecphi{{\boldsymbol \phi}}
\def\vecpsi{{\boldsymbol \psi}}
\def\vecsigma{{\mathbf \sigma}}
\def\vecvarsigma{{\mathbf \varsigma}}
\def\vectau{{\boldsymbol \tau}}
\def\vecupsilon{{\boldsymbol \upsilon}}
\def\vecvarphi{{\boldsymbol \varphi}}
\def\vecxi{{\boldsymbol \xi}}
\def\veczeta{{\boldsymbol \zeta}}


\def\vecX{{\mathbf X}}
\def\vecY{{\mathbf Y}}
\def\vecZ{{\mathbf Z}}

\def\vecEpsilon{{\mathbf \Epsilon}}
\def\vecEta{{\mathbf \Eta}}


\def\matA{{\mathbf A}}
\def\matB{{\mathbf B}}
\def\matC{{\mathbf C}}
\def\matD{{\mathbf D}}
\def\matE{{\mathbf E}}
\def\matF{{\mathbf F}}
\def\matH{{\mathbf H}}
\def\matI{{\mathbf I}}
\def\matK{{\mathbf K}}
\def\matL{{\mathbf L}}
\def\matO{{\mathbf O}}
\def\matOmega{{\mathbf \Omega}}
\def\matN{{\mathbf N}}
\def\matP{{\mathbf P}}
\def\matS{{\mathbf S}}
\def\matU{{\mathbf U}}
\def\matV{{\mathbf V}}
\def\matW{{\mathbf W}}
\def\matX{{\mathbf X}}
\def\matZ{{\mathbf Z}}
\def\matGamma{{\mathbf{\mathrm{\Gamma}}}}
\def\matLambda{{\mathbf{\mathrm{\Lambda}}}}
\def\matOmega{{\mathbf{\mathrm{\Omega}}}}
\def\matPhi{{\mathbf \Phi}}
\def\matPsi{{\mathbf \Psi}}
\def\matRho{{\mathbf{\mathrm{P}}}}
\def\matSigma{{\mathbf{\mathrm{\Sigma}}}}
\def\matUpsilon{{\mathbf{\mathrm{\Upsilon}}}}
\def\matXi{{\mathbf{\mathrm{\Xi}}}}

\def\matzero{{\mathbf 0}}


\def\complex{{\mathbb {C}}}
\def\real{{\mathbb {R}}}
\def\extreal{\overline{\mathbb {R}}}
\def\rational{{\mathbb {Q}}}
\def\pnint{{\mathbb {Z}}}
\def\nnint{{\mathbb{N}_0}}
\def\pint{{\mathbb {N}}}
\def\extint{\overline{\mathbb {Z}}}

\def\defas{:=}
\def\as{\overset{\mbox{a.s.}}{=}}
\def\ind{1}
\def\normal{\calN}
\def\expect{\mathbb{E}}
\def\variance{\mbox{Var}}
\def\covariance{\mbox{Cov}}
\def\prob{\mathbb{P}}
\def\risk{\calR}
\def\Uniform{{\cal{U}}}
\def\Trace{\mbox{Tr}}
\def\sign{\mbox{sign}}
\def\evaloss{\star}
\def\margin{\varrho}
\def\Log{{Log}}

\def\converged{\xrightarrow[]{D}}
\def\convergep{\xrightarrow[]{P}}
\def\convergeas{\xrightarrow[]{a.s.}}

\def\Borel{{\frakB}}

\def\Re{\mbox{Re}}
\def\Im{\mbox{Im}}

\def\interior{\mathrm{o}}

\def\calA{{\cal A}}
\def\calB{{\cal B}}
\def\calC{{\cal C}}
\def\calD{{\cal D}}
\def\calE{{\cal E}}
\def\calF{{\cal F}}
\def\calG{{\cal G}}
\def\calH{{\cal H}}
\def\calK{{\cal K}}
\def\calL{{\cal L}}
\def\calM{{\cal M}}
\def\calN{{\cal N}}
\def\calP{{\cal P}}
\def\calR{{\cal R}}
\def\calS{{\cal S}}
\def\calT{{\cal T}}
\def\calV{{\cal V}}
\def\calX{{\cal X}}
\def\calY{{\cal Y}}
\def\calZ{{\cal Z}}

\def\scrA{\mathscr{A}}
\def\scrB{\mathscr{B}}
\def\scrC{\mathscr{C}}
\def\scrD{\mathscr{D}}
\def\scrE{\mathscr{E}}
\def\scrF{\mathscr{F}}
\def\scrG{\mathscr{G}}
\def\scrI{\mathscr{I}}
\def\scrJ{\mathscr{J}}
\def\scrK{\mathscr{K}}
\def\scrL{\mathscr{L}}
\def\scrM{\mathscr{M}}
\def\scrN{\mathscr{N}}
\def\scrP{\mathscr{P}}
\def\scrQ{\mathscr{Q}}
\def\scrR{\mathscr{R}}
\def\scrS{\mathscr{S}}
\def\scrU{\mathscr{U}}
\def\scrV{\mathscr{V}}
\def\scrX{\mathscr{X}}


\def\pzch{\mathpzc{h}}
\def\pzcy{\mathpzc{y}}

\def\frakA{\mathfrak{A}}
\def\frakB{\mathfrak{B}}
\def\frakG{\mathfrak{G}}
\def\frakM{\mathfrak{M}}

\def\continuous{{\cal C}}


\addtolength{\oddsidemargin}{-.875in}
\addtolength{\evensidemargin}{-.875in}
\addtolength{\textwidth}{1.75in}
\addtolength{\topmargin}{-.875in}
\addtolength{\textheight}{1.75in}

\title{ML2023 Fall Homework Assignment 1\\ Handwritten}
\author{Lecturor: Pei-Yuan Wu\\
TAs: \textcolor{red}{Yuan-Chia Chang, Chun-Lin Huang}}
\date{Sep 2023}

\begin{document}

\maketitle



\section*{Problem 1 (Preliminary) (1 pt)}
\begin{enumerate}
    \item[(a)] (0.2 pts) Given $\vecw\in \real^m$,  $\matA\in\mathbb{R}^{m\times m}$. Show that
    $$\nabla_\vecw \vecw\tran \matA\mathbf{w} = \matA\tran \vecw + \matA\vecw.$$
    In particular, if $\matA$ is a symmetric matrix, then
    $$\nabla_\vecw \vecw\tran \matA\mathbf{w} = 2\matA\vecw$$
    %
    \item[(b)] (0.2 pts) Given $\matA, \matB \in \real^{m \times m}$. Show that
    \begin{equation}\label{eq:matrix_product_trace_derivative}
    \frac{\partial\ tr(\matA\matB)}{\partial a_{ij}} = b_{ji}
    \end{equation}
    where
    \begin{equation*}
    \matA=\left[\begin{array}{cccc}
    a_{11} & a_{12} & \cdots & a_{1 m} \\
    a_{21} & a_{22} & \cdots & a_{2 m} \\
    \vdots & \vdots & \ddots & \vdots \\
    a_{m 1} & a_{m 2} & \cdots & a_{m m}
    \end{array}\right], ~~
    \matB=\left[\begin{array}{cccc}
    b_{11} & b_{12} & \cdots & b_{1 m} \\
    b_{21} & b_{22} & \cdots & b_{2 m} \\
    \vdots & \vdots & \ddots & \vdots \\
    b_{m 1} & b_{m 2} & \cdots & b_{m m}
    \end{array}\right]
    \end{equation*}
    %
    It is common to write (\ref{eq:matrix_product_trace_derivative}) as     
    $$\frac{\partial\ tr(\matA\matB)}{\partial \matA} = \matB\tran.$$
    %
    \item[(c)] (0.6 pts) Prove that
    \begin{equation}\label{eq:matrix_det_log_derivative}
    \frac{\partial \log(\det(\boldsymbol{\matA}))}{\partial a_{ij}} =  \vece_j\tran \boldsymbol{\matA}^{-1}\space \vece_i,
    \end{equation}
    %
    where $\boldsymbol{\matA} = \left[\begin{array}{cccc}
    a_{11} & a_{12} & \cdots & a_{1 m} \\
    a_{21} & a_{22} & \cdots & a_{2 m} \\
    \vdots & \vdots & \ddots & \vdots \\
    a_{m 1} & a_{m 2} & \cdots & a_{m m}
    \end{array}\right]
     \in \real^{m\times m}$ is a (non-singular) matrix, and $\mathbf{e}_j$ is the unit vector along the j-th axis (e.g. $\mathbf{e}_3=[0,0,1,0,...,0]^T$).
     It is common to write (\ref{eq:matrix_det_log_derivative}) as
     $$\frac{\partial \log(\det(\matA))}{\partial \matA} =  \left(\matA^{-1}\right)\tran$$
\end{enumerate}


\section*{Problem 2 (Classification with Gaussian Mixture Model) (2.4 pts)}

In this question, we tackle the binary classification problem through the generative approach, where we assume the data point $X$ (viewed as a $\real^d$-valued r.v.) and its label $Y$ (viewed as a $\{\calC_1,\calC_2\}$-valued r.v.) are generated according to the generative model (paramerized by $\theta$) as follows:
\begin{equation}\label{eq:two_gaussian}
\prob_\theta [X = \vecx,Y = \calC_k] = \pi_k f_{\vecmu_k,\matSigma_k}(\vecx)~~ (k \in \{1,2\})
\end{equation}
%
where $\theta = (\pi_1,\pi_2,\vecmu_1,\vecmu_2,\matSigma_1,\matSigma_2)$ for which
\begin{equation*}
f_{\vecmu_k,\matSigma_k}(\vecx) = \frac{1}{(2\pi)^{d/2}}\frac{1}{\left| \matSigma_k \right|^{1/2}}\exp\left(-\frac{1}{2}(\vecx-\vecmu_k)\tran \matSigma_k^{-1}(\vecx-\vecmu_i)\right).
\end{equation*}
%
Now suppose we observe data points $\vecx_1,...,\vecx_N$ and their corresponding labels $y_1,...,y_N$.
\begin{enumerate}[label=(\alph*)]
\item (1.2 pt)
\begin{enumerate}[label=(\roman*)]
\item (0.3 pt) Please write down the likelihood function $L(\theta)$ that describes how likely the generative model would generate the observed data $\{(\vecx_i,y_i)\}_{i=1}^N$ in terms of $\theta = (\pi_1,\pi_2,\vecmu_1,\vecmu_2,\matSigma_1,\matSigma_2)$.
%
\item (0.3 pt) Find the maximum likelihood estimate $\theta^* = (\pi^*_1,\pi^*_2,\vecmu^*_1,\vecmu^*_2,\matSigma^*_1,\matSigma^*_2)$ that maximizes the likelihood function $L(\theta)$. %(Note that we do not assume $(\Sigma_1^*,\Sigma_2^*)$ are the same)
%
\item (0.3 pt) Write down $\prob_\theta[Y=\calC_1|X=\vecx]$ and $\prob_\theta[X=\vecx|Y=\calC_1]$ in terms of $\theta = (\pi_1,\pi_2,\vecmu_1,\vecmu_2,\matSigma_1,\matSigma_2)$. What are the physical meaning of the aforementioned quantities?
%
\item (0.3 pt) Express $\prob_\theta[X=\vecx|Y=\calC_1]$ in the form of $\sigma(z)$, where $\sigma(\cdot)$ denotes the sigmoid function, and express $z$ in terms of $\theta = (\pi_1,\pi_2,\vecmu_1,\vecmu_2,\matSigma_1,\matSigma_2)$ and $x$.
\end{enumerate}
%
\item (1.2 pt) Suppose we pose an additional constraint that the covariance matrices of the two Gaussian distributions are identical, namely $\Sigma_1=\Sigma_2=\Sigma$, in which the generative model is parameterized by $\vartheta = (\pi_1,\pi_2,\vecmu_1,\vecmu_2,\matSigma)$. Redo questions (a) under such setting.
\end{enumerate}


\section*{Problem 3 (Application of Gaussian Mixture Model Classifier) (0.6 pts)}
In this question, you will train a binary classifier based on the data which can be downloaded from \href{https://reurl.cc/2EZMzn}{https://reurl.cc/2EZMzn}  following the settings in Problem 2. Each data point and its label take the format $x_i \in \real^2$ and $y_i \in \{0,1\}$. 
\begin{enumerate}[label=(\alph*)]
    \item (0.2 pts) Calculate $\vartheta^* = (\pi^*_1,\pi^*_2,\vecmu^*_1,\vecmu^*_2,\matSigma^*)$ as in Problem 2 (b) in numbers.
    \item (0.2 pts) Calculate $\theta^* = (\pi^*_1,\pi^*_2,\vecmu^*_1,\vecmu^*_2,\matSigma^*_1,\matSigma^*_2)$ as in Problem 2 (a)(ii) in numbers.
    \item (0.2 pts) Please draw the scatter plot of the data. Which model is better in your opinion between (a) and (b)? Why?
\end{enumerate}
 

\section*{Problem 4 (Closed-Form Linear Regression Solution) (1 pts + Bonus 1.5 pts)}
Consider the linear regression model
$$
\mathbf{y} = \mathbf{X}\boldsymbol{\theta} + \boldsymbol{\epsilon},
$$
where $\mathbf{y}\in\mathbb{R}^n, \mathbf{X}\in\mathbb{R}^{n\times d}, \boldsymbol{\theta}\in\mathbb{R}^d$ and $\boldsymbol{\epsilon}\in\mathbb{R}^n$. Denote $\mathbf{X}_i\in \mathbb{R}^{1\times d}$ as the $i$-th row of $\mathbf{X}$, with the following interpretations:
\begin{itemize}
\item If the linear model has the bias term, then write $\boldsymbol{\theta} = [w_1, \cdots, w_m, b]\tran$ and $\mathbf{X}_i = [x_{i,1}, x_{i,2}, \cdots, x_{i,m}, 1]$, namely $d = m+1$.
\item If the linear model has no bias term, then write $\boldsymbol{\theta} = [w_1, \cdots, w_d]^T$ and $\mathbf{X}_i = [x_{i,1}, x_{i,2}, \cdots, x_{i,m}]$, namely $d = m$.
\end{itemize}
%
\begin{enumerate}[label=(\alph*)]
\item Without the bias term, consider the $L^2$-regularized loss function:
$$\sum_i \kappa_i \left(y_i-\boldsymbol{X}_{\mathrm{i}} \boldsymbol{\theta}\right)^2+\lambda \sum_j w_j^2, \ \lambda > 0.$$
Show that the optimal solution that minimizes the loss function is $\boldsymbol{\theta^*}$ = $\left( \boldsymbol{X}^T \boldsymbol{K} \boldsymbol{X} + \lambda \boldsymbol{I}\right)^{-1} \boldsymbol{X}^T \boldsymbol{K} \boldsymbol{y}$, where
\begin{equation*}
  \boldsymbol{K} =
  \begin{bmatrix}
    \kappa_{1} & & 0\\
    & \ddots & \\
    0 & & \kappa_{n}
  \end{bmatrix}
\end{equation*}
%
is a diagonal matrix and $\boldsymbol{I}$ is the $n \times n$ identical matrix. 
%
\item (Bonus, 1.5 pts) With the bias term, the $L^2$-regularized loss function becomes
$$\sum_i \kappa_i \left(y_i-\boldsymbol{X}_{\mathrm{i}} \boldsymbol{\theta} \right)^2+\lambda \sum_j w_j^2, \ \lambda > 0.$$

Show that the optimal solution that minimizes the loss function is $\boldsymbol{\theta^*} $ =  $[ \boldsymbol{w^{\star}}^T, b^{\star}]^T$, where 
\begin{gather*}
\boldsymbol{w^{\star}}=\left( \boldsymbol{\tilde{X}}^T \boldsymbol{K} \boldsymbol{\tilde{X}} + \lambda \boldsymbol{I} - \frac{1}{\Tr{(\boldsymbol{K})}} \boldsymbol{\tilde{X}}^T \boldsymbol{K} \boldsymbol{e} \boldsymbol{e}^T \boldsymbol{K} \boldsymbol{\tilde{X}} \right)^{-1} \boldsymbol{\tilde{X}}^T \boldsymbol{K} \left( \boldsymbol{y} - \frac{1}{\Tr{(\boldsymbol{K})}} \boldsymbol{e} \boldsymbol{e}^T \boldsymbol{K} \boldsymbol{y} \right),\\
b^{\star} = \frac{1}{\Tr{(\boldsymbol{K})}} \left( \boldsymbol{e}^T \boldsymbol{K} \boldsymbol{y} - \boldsymbol{e}^T \boldsymbol{K} \boldsymbol{\tilde{X}} \boldsymbol{w^{\star}} \right)^T
\end{gather*}
%
for which $\boldsymbol{e}= [1 \ ... \ 1]^T$ denotes the all one vector, $\boldsymbol{X} = [\boldsymbol{\tilde{X}} \boldsymbol{e}] $, $\Tr{(\boldsymbol{K})}$ is the trace of the matrix $\boldsymbol{K}$, and that $\matK$ and $\matI$ are defined as in (a).
\end{enumerate}

\section*{Problem 5 (Noise and Regularization) (1 pts)}
Consider the linear model $f_{\mathbf{w},b}: \mathbb{R}^k \rightarrow \mathbb{R}$, where $\mathbf{w} \in \mathbb{R}^k$ and $b \in \mathbb{R}$, defined as
$$f_{\mathbf{w},b}(x) = \mathbf{w}^T \mathbf{x} + b$$

Given dataset $S = \{(\mathbf{x}_i,y_i)\}_{i=1}^N$, if the inputs $\mathbf{x}_i \in \mathbb{R}^k$ are contaminated with input noise $\boldsymbol{\eta}_i \in \mathbb{R}^k$, we may consider the expected sum-of-squares loss in the presence of input noise as 
$${\tilde L}_{ss}(\mathbf{w},b) = \mathbb{E}\left[ \frac{1}{2N}\sum_{i=1}^{N}\left(f_{\mathbf{w},b}(\mathbf{x}_i + \boldsymbol{\eta}_i)-y_i\right)^2 \right]$$
where the expectation is taken over the randomness of input noises $\boldsymbol{\eta}_1,...,\boldsymbol{\eta}_N$. Additionally, the inputs ($\mathbf{x}_i$) and the input noise ($\boldsymbol{\eta}_i$) are independent.

Now assume the input noises $\mathbf{\eta}_i = [\eta_{i,1}, \eta_{i,2}, ... ,\eta_{i,k}]^T$ are random vectors with zero mean $\mathbb{E}[\eta_{i,j}] = 0$, and the covariance between components is given by
$$\mathbb{E}[\eta_{i,j}\eta_{i',j'}] = \delta_{i,i'}\delta_{j,j'} \sigma^2$$
where $\delta_{i,i'} = \left\{\begin{array}{ll} 
1 & \mbox{, if} ~ i = i'\\
0 & \mbox{, otherwise.}
\end{array}\right.$ denotes the Kronecker delta. 

Please show that
$${\tilde L}_{ss}(\mathbf{w},b) = \frac{1}{2N}\sum_{i=1}^{N}\left(f_{\mathbf{w},b}(\mathbf{x}_i)-y_i\right)^2 + \frac{\sigma^2}{2}\|\mathbf{w}\|^2$$

That is, minimizing the expected sum-of-squares loss in the presence of input noise is equivalent to minimizing noise-free sum-of-squares loss with the addition of a $L^2$-regularization term on the weights.
(Hint: $\|\bf{x}\|^2 = {\bf x}^T{\bf x} = tr({\bf x}{\bf x}\tran)$ and the square of a vector is dot product with itself)




\section*{Problem 6 (Mathematical Background) (0 pt)}
Please click the following link \url{https://www.cs.cmu.edu/~mgormley/courses/10601/homework/hw1.zip}
to download the Homework 1 from CMU 2023 Machine Learning Website. You are encouraged to practice Section 3 to Section 6 of this homework to brush up some of the mathematical background that will be useful for this course. \textbf{This problem will not be graded}. However, you are encouraged to consult TA by joining TA hour if you find any questions.

\section*{Some Tools You Need to Know}
\begin{enumerate}
\item Orthogonal Matrix
\item Positive Definite, Semipositive Definite
\item Eigenvalue Decomposition, Singular value decomposition
%\item Convexity
\item Lagrange Multiplier
\item Trace
\end{enumerate}
You can find the definition and the usage by yourself. It is also welcome to discuss with TA in TA hour.

%\newpage
%\textcolor{red}{As next homework. Give intermediate subproblems to hint how to progress}
%\section*{Problem XX (Gradient Descent Convergence)}
%Suppose the function $f : \mathbb{R}^n \rightarrow \mathbb{R}$
%is convex and differentiable, and that its gradient is
%Lipschitz continuous with constant $L > 0$, i.e. we have that $|| \nabla f(x) - \nabla f(y)||_2 \leq L||x - y||_2$ for any $x, y$.
%Then if we run gradient descent for $k$ iterations with a fixed step size $t \leq \frac{1}{L}$, it will yield a solution $x^{(k)}$ which satisfies
%$$f(x^{(k)})-f(x^{\star}) \leq \frac{||x^{(0)}-x^{*}||_2}{2tk}$$
%where $f(x^{\star})$ is the optimal value.

\end{document}
